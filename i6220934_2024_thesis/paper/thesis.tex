\documentclass[conference]{IEEEtran}
\IEEEoverridecommandlockouts
% The preceding line is only needed to identify funding in the first footnote. If that is unneeded, please comment it out.
\usepackage{cite}
\usepackage{amsmath,amssymb,amsfonts}
\usepackage{algorithmic}
\usepackage{graphicx}
\usepackage{textcomp}
\usepackage{tikz}
\usepackage{xcolor}
\def\BibTeX{{\rm B\kern-.05em{\sc i\kern-.025em b}\kern-.08em
    T\kern-.1667em\lower.7ex\hbox{E}\kern-.125emX}}
\begin{document}

\title{Exploring Tensor Decomposition In Time Series Prediction Models\\
\thanks{This thesis was prepared in partial fulfilment of the requirements for the Degree of Bachelor of Science in Data Science and Artificial Intelligence, Maastricht University. Supervisor(s): Philippe Dreesen}
}

\author{\IEEEauthorblockN{Noah Croes}
\IEEEauthorblockA{\textit{Department of Advanced Computing Sciences} \\
\textit{Faculty of Science and Engineering}\\
\textit{Maastricht University}\\
Maastricht, The Netherlands}
}

\maketitle

\begin{abstract}
hello
\end{abstract}

\begin{IEEEkeywords}
nuclear norm, data-driven simulation
\end{IEEEkeywords}

\section{Introduction}
The development of models with strong predictive capabilities is becoming increasingly more relevant, considering the vast amounts of data that is widely available. These models find its relevance not only in academic fields, but also in various other practical domains, including healthcare, finance, meteorology, and supply chain management. Specifically time-series prediction models aim to forecast future values or data points based off of historic data. The difficulty in creating such models lies in ensure the model fits the data appropriately. Historic data may demonstrate complex and highly variant patterns resulting in arising possible complications when designing predictive models due to over- and underfitting, noise, external events, data sparsity, etc. Hence, attempting to capture the inherent structure of the historic data as well as possible lies at the core of the development of time-series prediction models. 

Traditionally historic data is often represented in vector form



\section{Methods}
\section{Experiments}
\section{Results}
\section{Discussion}
\section{Conclusions}
\bibliography{sample}




\end{document}
